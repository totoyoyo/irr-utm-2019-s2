%stage 1
We were given the PDE
$$[\partial_t+\partial_{xxx}]q(x,t) = 0.$$
Perform a spatial Fourier Transform of the PDE, using our Preliminary Work, we get
\begin{align*}
    \widehat{[\partial_t+\partial_{xxx}]q}(x,t) &= \widehat{\partial_t q}(\lambda,t) + e^{-i\lambda x}\left(\partial_{xx}q(x,t) + i\lambda \partial_{x}q(x,t) - \lambda^2 q(x,t) \right) \bigg\rvert_0^1 - i \lambda^3\widehat{q}(\lambda,t) \\
    &= (\partial_t - i \lambda^3) \widehat{q}(\lambda,t) + e^{-i\lambda x}\left(\partial_{xx}q(x,t) + i\lambda \partial_{x}q(x,t) - \lambda^2 q(x,t) \right) \bigg\rvert_0^1 \\
    &= 0.
\end{align*}
Therefore, we have the ODE
\begin{align*}
&[\partial_t - i \lambda^3] \widehat{q}(\lambda,t) + e^{-i\lambda}\left(\partial_{xx}q(1,t) + i\lambda \partial_{x}q(1,t) - \lambda^2 q(1,t)\right) - \partial_{xx}q(0,t) - i\lambda \partial_{x}q(0,t) + \lambda^2 q(0,t) \\
&= 0.
\end{align*}
Multiply both sides by $e^{-i\lambda^3 t}$ to get
\begin{align*}
&e^{-i\lambda^3 t}[\partial_t - i \lambda^3] \widehat{q}(\lambda,t) + e^{-i\lambda -i\lambda^3 t}\left(\partial_{xx}q(1,t) + i\lambda \partial_{x}q(1,t) - \lambda^2 q(1,t)\right) \\
&- e^{-i\lambda^3 t}\left(\partial_{xx}q(0,t) + i\lambda \partial_{x}q(0,t) - \lambda^2 q(0,t)\right) \\
&= 0.
\end{align*}
Observe that
    $$e^{-i\lambda^3 t}[\partial_t - i \lambda^3] \widehat{q}(\lambda,t) = \frac{\mathrm{d}}{\mathrm{d}t}\left(e^{-i\lambda^3 t}\widehat{q}(\lambda,t)\right).$$
Integrate with respect to time to solve the ODE,
\begin{align*}
&\int_0^t \frac{\mathrm{d}}{\mathrm{d}s} \left(e^{-i\lambda^3 s} \widehat{q}(\lambda,s)\right) \mathrm{d}s \\
&-\int_0^t e^{-i\lambda^3 s}\left(\partial_{xx}q(0,s) + i\lambda \partial_{x}q(0,s) - \lambda^2 q(0,s)\right) \mathrm{d}s \\
&+\int_0^t e^{-i\lambda -i\lambda^3 s}\left(\partial_{xx}q(1,s) + i\lambda \partial_{x}q(1,s) - \lambda^2 q(1,s)\right) \mathrm{d}s \\
&= 0.
\end{align*}

We will introduce the notation
$$f_j(\lambda,X,t) := \int_0^t e^{-i\lambda^3 s}(\partial_x)^j q(X,s) \mathrm{d}s.$$
Therefore, we get
\begin{align*}
&e^{-i\lambda^3 t} \widehat{q}(\lambda,t) - \widehat{q}(\lambda,0) \\
&-\left(f_2(\lambda,0,t) + i\lambda f_1(\lambda,0,t) - \lambda^2 f_0(\lambda,0,t)\right) \\
&+e^{-i\lambda}\left(f_2(\lambda,1,t) + i\lambda f_1(\lambda,1,t) - \lambda^2 f_0(\lambda,1,t)\right) \\
&= 0
\end{align*}
which implies that
\begin{align*}
\widehat{q}(\lambda,t) = e^{i\lambda^3 t} \widehat{q}(\lambda,0) &+e^{i\lambda^3 t}\left(f_2(\lambda,0,t) + i\lambda f_1(\lambda,0,t) - \lambda^2 f_0(\lambda,0,t)\right) \\
&-e^{-i\lambda+i\lambda^3 t}\left(f_2(\lambda,1,t) + i\lambda f_1(\lambda,1,t) - \lambda^2 f_0(\lambda,1,t)\right)
\end{align*}
and this equation is the Global Relation (GR) which is valid for all $\lambda \in \mathbb{C}$ and for all $t \in [0,T]$. Take the inverse spatial Fourier Transform of GR and split the integrals on the left-hand side to get equation 6
\begin{align}
2\pi q(x,t) &= \int_{-\infty}^{\infty} e^{i\lambda x + i\lambda^3 t} \widehat{q}_0(\lambda) \mathrm{d}\lambda \label{eqn:IFT_GR}\\ 
&+ \int_{-\infty}^{\infty} e^{i\lambda (x-1)+i\lambda^3 t}\left(f_2(\lambda,1,t) + i\lambda f_1(\lambda,1,t) - \lambda^2 f_0(\lambda,1,t)\right) \mathrm{d}\lambda \nonumber \\
&- \int_{-\infty}^{\infty} e^{i\lambda x+i\lambda^3 t}\left(f_2(\lambda,0,t) + i\lambda f_1(\lambda,0,t) - \lambda^2 f_0(\lambda,0,t)\right) \mathrm{d}\lambda. \nonumber
\end{align}

We aim to deform the latter two contours of integration away from $\mathbb{R}$. Define
\begin{align*}
    &\mathbb{C}^{\pm} := \{\lambda \in \mathbb{C} \mid \mbox{Im}(\lambda) > 0 \} \\
    &D := \{ \lambda \in \mathbb{C} \mid \mbox{Re}(-i\lambda^3) < 0 \} &D^{\pm} := D \cap \mathbb{C}^{\pm} \\
    &E := \{ \lambda \in \mathbb{C} \mid \mbox{Re}(-i\lambda^3) > 0 \} &E^{\pm} := E \cap \mathbb{C}^{\pm}
\end{align*}
and orient the boundaries of these (unions of) sectors in the positive sense; the sector lies to the left of its boundary.
Note that if we let $\lambda = e^{i\theta}$ for some $\theta \in \mathbb{R}$, then we can re-write sets $D$ and $E$ as
\begin{align*}
    &D = \{ e^{i\theta} \in \mathbb{C} \mid \cos(3\theta - \frac{\pi}{2}) < 0 \} \\
    &E = \{ e^{i\theta} \in \mathbb{C} \mid \cos(3\theta - \frac{\pi}{2}) > 0 \}. \\
\end{align*}
Refer to figure \ref{fig:intcontour} for a diagram of the sectors $D^{\pm}$ and $E^{\pm}$ on the complex plane.
\begin{figure}
\includegraphics[width=\linewidth]{ImagesforFinalProblem/intcontour1.png}
\caption{A diagram showing the sectors $D^{\pm}$ and $E^{\pm}$ on the complex plane.}
\centering\label{fig:intcontour}
\end{figure}
\newpage Consider
$$f_j(\lambda,X,t) := \int_0^t e^{-i\lambda^3 s}(\partial_x)^j q(X,s) \mathrm{d}s,$$
multiply both sides by $e^{i\lambda^3 t}$ and integrate by parts in $s$, then
\begin{align*}
    e^{i\lambda^3 t}f_j(\lambda,X,t) &= \int_0^t e^{i\lambda^3 (t-s)}(\partial_x)^j q(X,s) \mathrm{d}s \\
    &= \frac{i}{i\lambda^3} e^{i\lambda^3 (t-s)}(\partial_x)^j q(X,s) \bigg\rvert_{s=0}^{s=t} - \frac{i}{\lambda^3}\int_0^t e^{i\lambda^3 (t-s)}(\partial_x)^j q(X,s) \mathrm{d}s. 
\end{align*}
$E$ was chosen so that Re($i\lambda^3(t-s)) \leq 0$ for all $s \in [0,t]$. Therefore, as $\lambda \rightarrow \infty$ within clos$(E)$, $e^{i\lambda^3 (t-s)} = \mathcal{O}(1)$ decays uniformly in arg$(\lambda)$, and by the Riemann-Lebesgue lemma, the integral in the second term is oscillatory. Therefore, $e^{i\lambda^3 t}f_j(\lambda,X,t)$ decays uniformly in arg$(\lambda)$ with $\mathcal{O}(\abs{\lambda^{-3}})$, which implies that the entire term
$$e^{i\lambda^3 t}\left(f_2(\lambda,0,t) + i\lambda f_1(\lambda,0,t) - \lambda^2 f_0(\lambda,0,t)\right) = \mathcal{O}(\abs{\lambda^{-1}}),$$
decays uniformly in arg$(\lambda)$ as $\lambda \rightarrow \infty$ within clos$(E)$.
Also, 
$$e^{i\lambda^3 t}\left(f_2(\lambda,0,t) + i\lambda f_1(\lambda,0,t) - \lambda^2 f_0(\lambda,0,t)\right) = \mathcal{O}(\abs{\lambda^{-1}})$$
is entire, i.e. analytic at all finite points of $\mathbb{C}$, as $(\partial_x)^jq(X,\cdot) \in L^1[0, T]$. Hence, by Jordan's lemma,
$$
    \int_{E^+} e^{i\lambda x+i\lambda^3 t}\left(f_2(\lambda,0,t) + i\lambda f_1(\lambda,0,t) - \lambda^2 f_0(\lambda,0,t)\right) \mathrm{d}\lambda = 0,
$$ 
and similarly,
$$
    \int_{E^-} e^{i\lambda (x-1)+i\lambda^3 t}\left(f_2(\lambda,1,t) + i\lambda f_1(\lambda,1,t) - \lambda^2 f_0(\lambda,1,t)\right) \mathrm{d}\lambda = 0.
$$
Temporarily denote
\begin{align*}
    &I_2 := e^{i\lambda (x-1)+i\lambda^3 t}\left(f_2(\lambda,1,t) + i\lambda f_1(\lambda,1,t) - \lambda^2 f_0(\lambda,1,t)\right), \\
    &I_3 := e^{i\lambda x+i\lambda^3 t}\left(f_2(\lambda,0,t) + i\lambda f_1(\lambda,0,t) - \lambda^2 f_0(\lambda,0,t)\right), \\
\end{align*} then,
\begin{align*}
    &\int_{-\infty}^{\infty} I_2 \mathrm{d}\lambda = -\int_{\infty}^{-\infty} I_2 \mathrm{d}\lambda = -\left\{\int_{\infty}^{-\infty} - \int_{E^-}\right\} I_2 \mathrm{d}\lambda = -\int_{D^-} I_2 \mathrm{d}\lambda \\
    &\int_{-\infty}^{\infty} I_3 \mathrm{d}\lambda = \left\{\int_{-\infty}^{\infty} - \int_{E^+}\right\} I_3 \mathrm{d}\lambda = \int_{D^+} I_3 \mathrm{d}\lambda. \\
\end{align*}
Therefore, we can re-write equation \ref{eqn:IFT_GR} as
\begin{align}
2\pi q(x,t) &= \int_{-\infty}^{\infty} e^{i\lambda x + i\lambda^3 t} \widehat{q}_0(\lambda) \mathrm{d}\lambda \label{eqn:EFt}\\ 
&- \int_{D^-} e^{i\lambda (x-1)+i\lambda^3 t}\left(f_2(\lambda,1,t) + i\lambda f_1(\lambda,1,t) - \lambda^2 f_0(\lambda,1,t)\right) \mathrm{d}\lambda \nonumber \\
&- \int_{D^+} e^{i\lambda x+i\lambda^3 t}\left(f_2(\lambda,0,t) + i\lambda f_1(\lambda,0,t) - \lambda^2 f_0(\lambda,0,t)\right) \mathrm{d}\lambda \nonumber
\end{align}
and this is the Ehrenpresis form (EFt) which is valid for $(x,t) \in (0,1) \times [0,T].$

By a similar argument, $\forall \tau \in [t,T],$ Re($i\lambda^3(t-s)) \leq 0$ within clos$(D)$ and therefore,
$$\int_t^\tau e^{i\lambda^3 (t-s)}\left(\partial_{xx}q(0,s) + i\lambda \partial_{x}q(0,s) - \lambda^2 q(0,s)\right) \mathrm{d}s$$
decays uniformly in arg$(\lambda)$ as $\lambda \rightarrow \infty$ within clos$(D)$.
Therefore, by Jordan's lemma, for sufficiently large $t$,
$$
    \int_{D^+} e^{i\lambda x+i\lambda^3 t}\left(f_2(\lambda,0,t) + i\lambda f_1(\lambda,0,t) - \lambda^2 f_0(\lambda,0,t)\right) \mathrm{d}\lambda = 0
$$
and
$$
    \int_{D^-} e^{i\lambda (x-1)+i\lambda^3 t}\left(f_2(\lambda,1,t) + i\lambda f_1(\lambda,1,t) - \lambda^2 f_0(\lambda,1,t)\right) \mathrm{d}\lambda = 0.
$$
This allows us to further manipulate equation \ref{eqn:EFt} to
\begin{align}
2\pi q(x,t) &= \int_{-\infty}^{\infty} e^{i\lambda x + i\lambda^3 t} \widehat{q}_0(\lambda) \mathrm{d}\lambda \label{eqn:EFtfinal}\\ 
&- \int_{D^-} e^{i\lambda (x-1)+i\lambda^3 t}\left(f_2(\lambda,1,t) + i\lambda f_1(\lambda,1,t) - \lambda^2 f_0(\lambda,1,t)\right) \mathrm{d}\lambda \nonumber \\
&- \int_{D^+} e^{i\lambda x+i\lambda^3 t}\left(f_2(\lambda,0,t) + i\lambda f_1(\lambda,0,t) - \lambda^2 f_0(\lambda,0,t)\right) \mathrm{d}\lambda \nonumber
\end{align}
which is valid for $(x,t) \in (0,1) \times [0,\tau], \tau \in [0,T]$ (in practice we will pick $\tau$ to be larger than all the time-values we care about).
This formula (EF$\tau$) has the advantage of very simple $(x,t)$ dependence.

We started by assuming that a solution exists and showed that any solution must satisfy (EF$\tau$) and (GR). 
The importance of the equation (GR) is not yet clear, except for deriving (EF$\tau$). 
Moreover, (EF$\tau$) is not an explicit representation of the solution since
\begin{enumerate}
    \item $f_1(\lambda,0,t)$
    \item $f_2(\lambda,0,t)$
    \item $f_2(\lambda,1,t)$
\end{enumerate}
are unknown.